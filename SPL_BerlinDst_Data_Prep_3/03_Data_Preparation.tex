\documentclass[]{article}
\usepackage{lmodern}
\usepackage{amssymb,amsmath}
\usepackage{ifxetex,ifluatex}
\usepackage{fixltx2e} % provides \textsubscript
\ifnum 0\ifxetex 1\fi\ifluatex 1\fi=0 % if pdftex
  \usepackage[T1]{fontenc}
  \usepackage[utf8]{inputenc}
\else % if luatex or xelatex
  \ifxetex
    \usepackage{mathspec}
  \else
    \usepackage{fontspec}
  \fi
  \defaultfontfeatures{Ligatures=TeX,Scale=MatchLowercase}
\fi
% use upquote if available, for straight quotes in verbatim environments
\IfFileExists{upquote.sty}{\usepackage{upquote}}{}
% use microtype if available
\IfFileExists{microtype.sty}{%
\usepackage{microtype}
\UseMicrotypeSet[protrusion]{basicmath} % disable protrusion for tt fonts
}{}
\usepackage[margin=1in]{geometry}
\usepackage{hyperref}
\hypersetup{unicode=true,
            pdftitle={03\_Data\_Preparation},
            pdfauthor={Aleksandra Kudaeva},
            pdfborder={0 0 0},
            breaklinks=true}
\urlstyle{same}  % don't use monospace font for urls
\usepackage{color}
\usepackage{fancyvrb}
\newcommand{\VerbBar}{|}
\newcommand{\VERB}{\Verb[commandchars=\\\{\}]}
\DefineVerbatimEnvironment{Highlighting}{Verbatim}{commandchars=\\\{\}}
% Add ',fontsize=\small' for more characters per line
\usepackage{framed}
\definecolor{shadecolor}{RGB}{248,248,248}
\newenvironment{Shaded}{\begin{snugshade}}{\end{snugshade}}
\newcommand{\KeywordTok}[1]{\textcolor[rgb]{0.13,0.29,0.53}{\textbf{#1}}}
\newcommand{\DataTypeTok}[1]{\textcolor[rgb]{0.13,0.29,0.53}{#1}}
\newcommand{\DecValTok}[1]{\textcolor[rgb]{0.00,0.00,0.81}{#1}}
\newcommand{\BaseNTok}[1]{\textcolor[rgb]{0.00,0.00,0.81}{#1}}
\newcommand{\FloatTok}[1]{\textcolor[rgb]{0.00,0.00,0.81}{#1}}
\newcommand{\ConstantTok}[1]{\textcolor[rgb]{0.00,0.00,0.00}{#1}}
\newcommand{\CharTok}[1]{\textcolor[rgb]{0.31,0.60,0.02}{#1}}
\newcommand{\SpecialCharTok}[1]{\textcolor[rgb]{0.00,0.00,0.00}{#1}}
\newcommand{\StringTok}[1]{\textcolor[rgb]{0.31,0.60,0.02}{#1}}
\newcommand{\VerbatimStringTok}[1]{\textcolor[rgb]{0.31,0.60,0.02}{#1}}
\newcommand{\SpecialStringTok}[1]{\textcolor[rgb]{0.31,0.60,0.02}{#1}}
\newcommand{\ImportTok}[1]{#1}
\newcommand{\CommentTok}[1]{\textcolor[rgb]{0.56,0.35,0.01}{\textit{#1}}}
\newcommand{\DocumentationTok}[1]{\textcolor[rgb]{0.56,0.35,0.01}{\textbf{\textit{#1}}}}
\newcommand{\AnnotationTok}[1]{\textcolor[rgb]{0.56,0.35,0.01}{\textbf{\textit{#1}}}}
\newcommand{\CommentVarTok}[1]{\textcolor[rgb]{0.56,0.35,0.01}{\textbf{\textit{#1}}}}
\newcommand{\OtherTok}[1]{\textcolor[rgb]{0.56,0.35,0.01}{#1}}
\newcommand{\FunctionTok}[1]{\textcolor[rgb]{0.00,0.00,0.00}{#1}}
\newcommand{\VariableTok}[1]{\textcolor[rgb]{0.00,0.00,0.00}{#1}}
\newcommand{\ControlFlowTok}[1]{\textcolor[rgb]{0.13,0.29,0.53}{\textbf{#1}}}
\newcommand{\OperatorTok}[1]{\textcolor[rgb]{0.81,0.36,0.00}{\textbf{#1}}}
\newcommand{\BuiltInTok}[1]{#1}
\newcommand{\ExtensionTok}[1]{#1}
\newcommand{\PreprocessorTok}[1]{\textcolor[rgb]{0.56,0.35,0.01}{\textit{#1}}}
\newcommand{\AttributeTok}[1]{\textcolor[rgb]{0.77,0.63,0.00}{#1}}
\newcommand{\RegionMarkerTok}[1]{#1}
\newcommand{\InformationTok}[1]{\textcolor[rgb]{0.56,0.35,0.01}{\textbf{\textit{#1}}}}
\newcommand{\WarningTok}[1]{\textcolor[rgb]{0.56,0.35,0.01}{\textbf{\textit{#1}}}}
\newcommand{\AlertTok}[1]{\textcolor[rgb]{0.94,0.16,0.16}{#1}}
\newcommand{\ErrorTok}[1]{\textcolor[rgb]{0.64,0.00,0.00}{\textbf{#1}}}
\newcommand{\NormalTok}[1]{#1}
\usepackage{graphicx,grffile}
\makeatletter
\def\maxwidth{\ifdim\Gin@nat@width>\linewidth\linewidth\else\Gin@nat@width\fi}
\def\maxheight{\ifdim\Gin@nat@height>\textheight\textheight\else\Gin@nat@height\fi}
\makeatother
% Scale images if necessary, so that they will not overflow the page
% margins by default, and it is still possible to overwrite the defaults
% using explicit options in \includegraphics[width, height, ...]{}
\setkeys{Gin}{width=\maxwidth,height=\maxheight,keepaspectratio}
\IfFileExists{parskip.sty}{%
\usepackage{parskip}
}{% else
\setlength{\parindent}{0pt}
\setlength{\parskip}{6pt plus 2pt minus 1pt}
}
\setlength{\emergencystretch}{3em}  % prevent overfull lines
\providecommand{\tightlist}{%
  \setlength{\itemsep}{0pt}\setlength{\parskip}{0pt}}
\setcounter{secnumdepth}{0}
% Redefines (sub)paragraphs to behave more like sections
\ifx\paragraph\undefined\else
\let\oldparagraph\paragraph
\renewcommand{\paragraph}[1]{\oldparagraph{#1}\mbox{}}
\fi
\ifx\subparagraph\undefined\else
\let\oldsubparagraph\subparagraph
\renewcommand{\subparagraph}[1]{\oldsubparagraph{#1}\mbox{}}
\fi

%%% Use protect on footnotes to avoid problems with footnotes in titles
\let\rmarkdownfootnote\footnote%
\def\footnote{\protect\rmarkdownfootnote}

%%% Change title format to be more compact
\usepackage{titling}

% Create subtitle command for use in maketitle
\newcommand{\subtitle}[1]{
  \posttitle{
    \begin{center}\large#1\end{center}
    }
}

\setlength{\droptitle}{-2em}

  \title{03\_Data\_Preparation}
    \pretitle{\vspace{\droptitle}\centering\huge}
  \posttitle{\par}
    \author{Aleksandra Kudaeva}
    \preauthor{\centering\large\emph}
  \postauthor{\par}
      \predate{\centering\large\emph}
  \postdate{\par}
    \date{28 Februar 2019}


\begin{document}
\maketitle

\subsection{Preparation of Air-Pollution
Data}\label{preparation-of-air-pollution-data}

Preparation of air-pollution data required obtaining additional data and
usage of several approximations. The whole process of data preparation
can be devided in three parts: 1) downloading the data from different
sources 2) data cleaning and reformating 3) merging and aggregating the
data

\subsubsection{Loading the data}\label{loading-the-data}

Air-pollution data was downloaded from {[}*insert reference:
``\url{https://fbinter.stadt-berlin.de/fb/index.jsp\%22*}{]} in xls
format. Initial column names are poorly adapted for R processing (e.g.
``PM10-Belastung (berechnetes Jahresmittel {[}µg/m³{]}) 2015'') as they
are too long and contain special symbols. For the purpose of further
analysis we rename all the columns according to the correspondence
stated in a matching table (matching.csv)

\begin{Shaded}
\begin{Highlighting}[]
\CommentTok{# Read air-pollution data from excel}
\NormalTok{ap15 =}\StringTok{ }\KeywordTok{read_excel}\NormalTok{(}\StringTok{"./SPL_BerlinDst_Data_Prep_3/Air_Pollution_2015.xls"}\NormalTok{, }
                  \DataTypeTok{sheet =} \DecValTok{1}\NormalTok{)}

\CommentTok{# Original names are too long and contain special symbols and spaces}
\NormalTok{mtch =}\StringTok{ }\KeywordTok{read.csv2}\NormalTok{(}\StringTok{"./SPL_BerlinDst_Data_Prep_3/matching.csv"}\NormalTok{, }
                 \DataTypeTok{sep =} \StringTok{";"}\NormalTok{, }
                 \DataTypeTok{stringsAsFactors =} \OtherTok{FALSE}\NormalTok{)  }\CommentTok{# Matching table for short names}

\KeywordTok{names}\NormalTok{(ap15) =}\StringTok{ }\NormalTok{mtch}\OperatorTok{$}\NormalTok{new[}\KeywordTok{match}\NormalTok{(}\KeywordTok{names}\NormalTok{(ap15), mtch}\OperatorTok{$}\NormalTok{old)]  }\CommentTok{# Rename variables}
\end{Highlighting}
\end{Shaded}

Downloaded table consists of 12374 rows containing values of different
air-pollution indicators and aggregated index values for section of main
Berlin Streets (1238). In our research we use PM10 and PM25 indicators
because those ones are regulated by European Union and have clearly
determined thresholds {[}\emph{insert reference to guidelines}{]}.
Average value of PM10 pollution is equal to 20.6 mg/m\^{}3, and PM25 -
14,3 (on average in 2015), which is below dangerous threshold.
Statistics were calculated by means of the script provided below:

\begin{Shaded}
\begin{Highlighting}[]
\CommentTok{# descriptory statistics of air pollution data table}
\NormalTok{ap15 }\OperatorTok\StringTok{ }\KeywordTok{summarize}\NormalTok{(}\DataTypeTok{Sections =} \KeywordTok{n}\NormalTok{(),                 }\CommentTok{# Number of street sections}
                   \DataTypeTok{Streets  =} \KeywordTok{n_distinct}\NormalTok{(Street),  }\CommentTok{# Number of unique streets}
                   \DataTypeTok{avg_PM10 =} \KeywordTok{mean}\NormalTok{(PM10_yearly),   }\CommentTok{# Average value of PM10}
                   \DataTypeTok{avg_PM25 =} \KeywordTok{mean}\NormalTok{(PM25_yearly))   }\CommentTok{# Average value of PM25}
\end{Highlighting}
\end{Shaded}

The raw data does not contain district key. That is why, in order to
bring Air-Pollution data to the same format as other variables described
in previous chapters and merge tables we need to download additional
information. Our solution was to scrap correspondence table from
web-page {[}\emph{insert reference: kauperts}{]}:

\begin{Shaded}
\begin{Highlighting}[]
\ControlFlowTok{for}\NormalTok{ (i }\ControlFlowTok{in} \DecValTok{1}\OperatorTok{:}\KeywordTok{dim}\NormalTok{(dstr)[}\DecValTok{1}\NormalTok{]) \{}
    \CommentTok{# Generate a link to data for all the districts and sub-districts}
\NormalTok{    link=}\KeywordTok{paste0}\NormalTok{(}\StringTok{"https://berlin.kauperts.de/Bezirke/"}\NormalTok{,}
\NormalTok{                dstr}\OperatorTok{$}\NormalTok{District[i],}
                \StringTok{"/Ortsteile/"}\NormalTok{,}
\NormalTok{                dstr}\OperatorTok{$}\NormalTok{Sub.district[i],}
                \StringTok{"/Strassen"}\NormalTok{)}
    
    \CommentTok{# Download the data from the web-page with generated link}
\NormalTok{    webpage =}\StringTok{ }\KeywordTok{read_html}\NormalTok{(link)}
\NormalTok{    tbls    =}\StringTok{ }\KeywordTok{html_nodes}\NormalTok{(webpage, }\StringTok{"table"}\NormalTok{)}
\NormalTok{    tab     =}\StringTok{ }\KeywordTok{html_table}\NormalTok{(tbls)[[}\DecValTok{1}\NormalTok{]]}
    
    \CommentTok{# Add columns for district and sub-district}
\NormalTok{    tab}\OperatorTok{$}\NormalTok{District    =}\StringTok{ }\NormalTok{dstr}\OperatorTok{$}\NormalTok{District[i]}
\NormalTok{    tab}\OperatorTok{$}\NormalTok{SubDistrict =}\StringTok{ }\NormalTok{dstr}\OperatorTok{$}\NormalTok{Sub.district[i]}
    
    \CommentTok{# Add created table to the table for the whole Berlin}
\NormalTok{    StrMtch =}\StringTok{ }\KeywordTok{rbind}\NormalTok{(StrMtch, tab)}
\NormalTok{\}}
\end{Highlighting}
\end{Shaded}

Downloaded table contains information on XXX streets from 12 districts
of Berlin.

\subsubsection{Reformating and cleaning}\label{reformating-and-cleaning}

Another problem was different ways of writing street name,
e.g.~Adamstr./ Africanische Str. and so on. First step to unification of
street names was replacements of german special symbols and bringing
everything to the low case. For that purpose the function
\emph{ReplaceUmlauts} was written:

\begin{Shaded}
\begin{Highlighting}[]
\CommentTok{#replace all the Umlauts by latin equivalents}
\NormalTok{ReplaceUmlauts =}\StringTok{ }\ControlFlowTok{function}\NormalTok{(clmn)\{}
    \CommentTok{#Description: Replaces german special symbols and turns to lower case}
    \CommentTok{#Author: Aleksandra Kudaeva}
    \CommentTok{#Input:  column where you want to replace umlauts}
    \CommentTok{#Output: column without umlauts (lower case)}
    
\NormalTok{    clmn =}\StringTok{ }\KeywordTok{tolower}\NormalTok{(clmn)  }\CommentTok{#all strings to lower case}
    
    \CommentTok{#check if at least one element of a vector has any umlauts in it}
    \CommentTok{#replaces umlauts until there are no one left}
    \ControlFlowTok{while}\NormalTok{(}\KeywordTok{any}\NormalTok{(}\KeywordTok{grepl}\NormalTok{(}\StringTok{"ä|ö|ü|ß"}\NormalTok{,clmn)) }\OperatorTok{==}\StringTok{ }\OtherTok{TRUE}\NormalTok{) \{}
\NormalTok{        clmn }\OperatorTok\StringTok{ }
\StringTok{            }\KeywordTok{sub}\NormalTok{(}\StringTok{"ä"}\NormalTok{, }\StringTok{"ae"}\NormalTok{, .) }\OperatorTok\StringTok{ }
\StringTok{            }\KeywordTok{sub}\NormalTok{(}\StringTok{"ö"}\NormalTok{, }\StringTok{"oe"}\NormalTok{, .) }\OperatorTok\StringTok{ }
\StringTok{            }\KeywordTok{sub}\NormalTok{(}\StringTok{"ü"}\NormalTok{, }\StringTok{"ue"}\NormalTok{, .) }\OperatorTok\StringTok{ }
\StringTok{            }\KeywordTok{sub}\NormalTok{(}\StringTok{"ß"}\NormalTok{, }\StringTok{"ss"}\NormalTok{, .)}
\NormalTok{    \}}
    \KeywordTok{return}\NormalTok{(clmn)}
\NormalTok{\}}
\end{Highlighting}
\end{Shaded}

Next step, was to substract unique part of street name (e.g.~Adamstr. -
Adam). THe following code illustrates work of replacement function and
substitution procedure for air-pollution dataset (analogical procedure
was performed for correspondence table):

\begin{Shaded}
\begin{Highlighting}[]
\CommentTok{# Street name formatting (in order to merge with street-index matching table)}
\NormalTok{ap15}\OperatorTok{$}\NormalTok{str =}\StringTok{ }\NormalTok{ap15}\OperatorTok{$}\NormalTok{Street }\OperatorTok
\StringTok{    }\KeywordTok{ReplaceUmlauts}\NormalTok{() }\OperatorTok\StringTok{  }\CommentTok{# Replace umlauts and switch to lower case}
\StringTok{    }\KeywordTok{sub}\NormalTok{(}\StringTok{"str.$|str$|-strasse|strasse|-str.$"}\NormalTok{, }\StringTok{""}\NormalTok{, .) }\OperatorTok\StringTok{  }\CommentTok{# Delete street ind.}
\StringTok{    }\KeywordTok{sub}\NormalTok{(}\StringTok{"ak |as |ad "}\NormalTok{, }\StringTok{""}\NormalTok{, .)  }\CommentTok{# Delete AK, AS, AD in the beginning}
\end{Highlighting}
\end{Shaded}

\subsubsection{Merging and summarizing}\label{merging-and-summarizing}

First step is\ldots{}. After we unified street names, tables are ready
to be merged by street name


\end{document}
